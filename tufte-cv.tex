\documentclass{tufte-handout}
\usepackage[usenames,dvipsnames,svgnames,table]{xcolor}
\usepackage{indentfirst}
\title{Tom Wallis}
\date{}
\begin{document}
\maketitle

\begin{abstract}
	\marginnote{cv@tomwallis.net\\
	\noindent \url{http://tomwallis.net/}\\
\noindent Phone: 07704825401\\
\noindent Address: 2/1 48 Murano Street\\ Glasgow\\ G20 7RU}
I'm a computing science student from Glasgow University with an appetite for unusual solutions to complex problems.\\
I like to do this by tying the knots between different subjects. This involves marrying a deep understanding of informatics and a breadth of knowledge from arts and sciences to come at problems from new angles. As a result, I work with systems modelling and algorithmics, as well as linguistics, culture, design and productivity studies.
\end{abstract}

\section{Computing Science}
Most computing science students have similar skills, like a fluency in languages such as Python, C, and Java. As well as standard taught languages, I have a solid foundation in \LaTeX, Javascript {\&} web technologies, Haskell, and practices like Agile and TDD.\marginnote{\url{https://github.com/probablytom}}\\
However, the cream of the crop are set apart by an understanding of how to \emph{use} their skills. In addition to participating in a university hackathon, my dissertation explores my interest in computers and culture by modelling sociotechnical systems and assessing risk\marginnote{\url{https://github.com/probablytom/behaviour-modelling}}. My third year dissertation involved team organisation through a novel project management and ticketing system\marginnote{\url{https://github.com/darrenburns/TeamProject3}}. As a hobby, I have applied software engineering tools like design patterns to improvised storytelling\marginnote{\url{http://projectalbert.net/}}, with great success. 

\section{Further Skills}
With any ordinary graduate, one would expect at least some degree of proficiency in their area of study. An agile mind, however, learns all the time about a wealth of subjects. I have a curiosity in maths and physics, having studied both at university level. In my spare time, I gravitate toward learning about linguistics and design.\\
A particular interest is typography and text-oriented design; this CV is designed with Matthew Butterick\marginnote{\url{http://practicaltypography.com/}} and Edward Tufte in mind. \\
Both as an employee of Glasgow University and as an optional part of my courses, I have taught in both schools and university lab sessions. I have learned to explain ideas intuitively and at a range of ability levels, and I'm excited to share what I know with others.

\section{Other Interests}
I enjoy music, podcasts, speciality coffee, fountain pens, and terrible movies when I'm not creating or learning. I'm always curious about etymologies, and I write sporadically about all of this at my blog. \marginnote{\url{http://blog.tomwallis.net}}

\end{document}
