\documentclass[a4paper,textheight=740px]{tufte-handout}
\usepackage[usenames,dvipsnames,svgnames,table]{xcolor}
\usepackage{indentfirst}
\title{Tom Wallis\hspace{20pt}\footnotesize{MSci Computing Science, Glasgow University}}}
\date{}

\newcommand{\projectlink}[1]{\marginnote{\url{#1}}}

\begin{document}
\vspace{-1cm}
\maketitle

\begin{abstract}
	\marginnote{
    cv@tomwallis.net\\
	\noindent \url{http://tomwallis.net/}\\
\noindent Phone: 07704825401\\
\noindent Address: 1/2 29 Crathie Drive\\ Glasgow\\ G11 7XE }
I'm a pragmatic worker who enjoys actively solving problems. Where many people
from a strong academic background drift toward theoretical and heady
approaches in a narrow field, I enjoy getting my hands dirty with
practical work, drawing on an enthusiasm for fields in STEM, social sciences and
humanities to attack interesting problems in genuinely new ways.
% It's not enough
% to pursue new knowledge: I aim to put our academic discoveries to use in the
% service of making a better world.

\end{abstract}

\section{Research}
I'm a first year computing science PhD candidate from Glasgow University,
studying new methods for modelling contingent behaviour (especially in
sociotechnical systems).\par

My research develops new ways to simluate variations in system behaviour when
modelling, alongside my supervisor, Tim Storer. We are currently developing a
series of libraries to do so, including
ASP\projectlink{http://github.com/probablytom/asp}, an aspect-oriented
programming (AOP) library encouraging separation of concerns in AOP,
PyDySoFu\projectlink{http://github.com/probablytom/pydysofu}, a library which
exploits ASP's design to implement process fuzzing, and Fuzzi
Moss\projectlink{http://github.com/twsswt/fuzzi-moss}, which employs process
fuzzing to inject human-like faults into the behaviours of sociotechnical
agents. A preliminary paper on the research is due to be published in the CAISE
forum 2018.


\section{Education}
\vspace{-0.5cm}
\begin{figure}
  \begin{adjustwidth}{-0.5cm}{}
    \begin{tabular*}{\paperwidth-5cm}{r||@{\extracolsep{\fill}}rll}
      \emph{Time} & \emph{Institution} & \emph{Qualification} & \emph{Grade}\\\hline\hline
      2017 --- 2021 & Glasgow University & Ph.D Computing Science & Ongoing\\
      2012 --- 2017 & Glasgow University & MSci Computing Science & B1 Average (2:1)\\
      2006 --- 2012 & St. Mungos High School & \emph{N/A} & \begin{tabular}{l} AAB at Advanced
                                                 Higher\\AAAABB at
                                                 Higher\end{tabular}
    \end{tabular*}
  \end{adjustwidth}
\end{figure}

\section{Employment}
\vspace{-0.5cm}
\begin{figure}
  \begin{adjustwidth}{-0.5cm}{}
    \begin{tabular*}{\paperwidth-5cm}{r||rl@{\extracolsep{\fill}}l}
      \emph{Time} & \emph{Workplace} & \emph{Job Title} & \emph{Reference Contact}\\\hline\hline
      Jan 2018 --- Present & Glasgow University & Demonstrator \& Tutor & \\
      May 2017 --- Sep 2017 & Obashi Tech & Software Engineer & \href{mailto:pjw@obashitech.com}{pjw@obashitech.com} \\
      Sep 2016 --- May 2017 & Self-Employed & ERP \& Python Consultant & \\
      June 2016 --- Sep 2016 & Gardenia Tech & Software Engineer & \href{mailto:contact@gardeniatech.com}{contact@gardeniatech.com}
    \end{tabular*}
  \end{adjustwidth}
\end{figure}

\vspace{0.5cm}

These positions have variously tested my ability to learn adaptively, work
strongly both in teams and on my own, and to take leading positions on large and
important projects. Employers have found me to be an effective and flexible
worker, but more importantly, a genuine asset to their teams in both individual
and collaborative environments. To demonstrate: my main client as a
self-employed consultant was my previous employer, Gardenia Technologies, and my
subsequent employer, Obashi Technologies, now parially fund my postgraduate
studies.

\end{document}
